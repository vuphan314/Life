\documentclass{paper}

\begin{document}

\versions

Revising my paper before submitting for publication.

\bigskip

\begin{center}

{\Large \mytitle} \\ \medskip
\vp \\
\textit{\ttu}

\end{center}

\bigskip

\begin{abstract}

We study syntactic conditions which guarantee when
a \cplong{} program has antichain property
(no answer set is a proper subset of another).
A notable such condition is that
the program's dependency graph being acyclic
and having no directed path from one cr-rule head literal
to another.

\end{abstract}

\bigskip

\tableofcontents

\newpage

\begin{flushleft}

%%%%%%%%%%%%%%%%%%%%%%%%%%%%%%%%%%%%%%%%%%%%%%%%%%%%%%%%%%%%

\section{Introduction}

\aplong{} (\ap) is a programming language
for knowledge representation and reasoning.
Every \ap{} program has antichain property:
no answer set is a proper subset of another.
This property is desirable,
because we want minimal solutions
to planning and diagnostic problems.

\bigskip

\cplong{} (\cp) extends \ap{} with cr-rules.
In \cp, indirect exceptions can be elegantly encoded.
However, there exist \cp{} programs
without antichain property.
We investigate some syntactic conditions
guaranteeing that a \cp{} program
has this desired semantic property.
Such conditions are specified in Theorem \ref{last_theorem}
(most notable) as well as
Theorem \ref{antichain_a_prolog},
Corollary \ref{one_cr_rule},
and Propositions \ref{unique_cr_literals} \&
\ref{same_abductive_support_heads}.

%%%%%%%%%%%%%%%%%%%%%%%%%%%%%%%%%%%%%%%%%%%%%%%%%%%%%%%%%%%%

\section{Preliminary Results}

\subsection{Background}

Note: the complete specifications of \ap{} and \cp{}
can be found in \cite[Sections 2.1, 2.2, 5.5]{krb}.
We focus on the \cp{} variant in which
minimality of abductive supports is based on cardinality.

\begin{definition}
[Regular Rule]

A \textdef{regular rule} has the form:
$$\regrule$$
where each $\lit_\ind$ is a ground literal and
$0 \le \nhl \le \nbl \le \nel$.

\end{definition}

\begin{definition}
[CR-Rule]

Similar to a regular rule,
a \textdef{cr-rule} has the form:
$$l_0 \whencr
l_1, \ldots, l_\nbl,\
not\ l_{\nbl + 1}, \ldots, not\ l_\nel.$$

\end{definition}

\begin{definition}
[\ap{} Program]

An \textdef{\ap{} program} is a finite set of
regular rules.

\end{definition}

\begin{definition}
[\cp{} Program]

A \textdef{\cp{} program} is a finite set of
regular rules and cr-rules.
Notations: $\preg$ and $\pcr$ denote
the corresponding subsets of regular rules and cr-rules.

\end{definition}

\begin{definition}
[Applications of CR-Rules]

Let $\rl$ be a cr-rule.
Then the \textdef{application} $\appl\rl$ is
the regular rule obtained from $\rl$ by
replacing $\whencr$ with $\when$.
If $\rls$ is a set of cr-rules, then let
$\appl\rls = \set{\appl\rl : \rl \in \rls}$.
When $\rls$ is a subset of a \cp{} program $\p$,
let $\pappld\rls$.

\end{definition}

\begin{definition}
[Antichain Property]

Let $\p$ be a \cp{} program.
We say $\p$ has \textdef{antichain property} if:
for all answer sets $\n\lits1$ and $\n\lits2$ of $\p$,
we have $\n\lits1 \sets \n\lits2 \impl \n\lits1 = \n\lits2$.

\end{definition}

\begin{example}
[\cp{} Program without Antichain Property]

Consider the following program:
\begin{gather*}
a. \\
\neg a \when \negd{b}, \negd{c}. \\
b \when c. \\
b \whencr. \tag{$\n\rl1$} \\
c \whencr. \tag{$\n\rl2$}
\end{gather*} \\
This program has answer sets $\n\lits1 = \{a, b\}$ and
$\n\lits2 = \{a, b, c\}$.
Observe that $\n\lits1 \setsp \n\lits2$ (proper subset).
The corresponding abductive supports are
$\n\rls1 = \set{\n\rl1}$ and $\n\rls2 = \set{\n\rl2}$.

\end{example}

\subsection{Up to One CR-Rule}

\begin{remark}
\label{vacuous_antichain}

Every inconsistent \cp{} program vacuously has
antichain property.
So only consistent programs are of interest.

\end{remark}

\begin{definition}
[Reduct]

Let $\rls$ be a set of regular rules and $\lits$ be a
consistent set of literals.
We write $\reduct{\rls}{\lits}$ for the \textdef{reduct} of
$\rls$ wrt $\lits$.

\end{definition}

\begin{lemma}
\label{reducts_subset}

Let $\p$ be an \ap{} program.
Let $\n\lits1$ and $\n\lits2$ be consistent sets of
literals.
If $\n\lits1 \sets \n\lits2$, then
$\reduct{\p}{\n\lits2} \sets \reduct{\p}{\n\lits1}$.

\end{lemma}

\begin{proof}

Assume $\reduct{\p}{\n\lits2}$ has an arbitrary rule $\rl$:
$$\regrulenodefneg$$
The corresponding rule in $\p$ is:
$$\regrule$$
For each $\ind$ in $\set{\nbl + 1, \ldots, \nel}$, we have
$l_\ind \notin \n\lits2$,
so $l_\ind \notin \n\lits1$ (as $\n\lits1 \sets \n\lits2$).
Therefore, $\rl$ is also a rule in $\reduct{\p}{\n\lits1}$.

\end{proof}

%:antichain A-Prolog
\begin{theorem}
[Antichain Property of \ap{} Programs, proved by \mg]
\label{antichain_a_prolog}

Let $\p$ be an \ap{} program.
Let $\n\lits1 \sets \n\lits2$ be answer sets of $\p$.
Then $\n\lits1 = \n\lits2$.

\end{theorem}

\begin{proof}

We will show $\n\lits2 \sets \n\lits1$.

\begin{enumerate}
\item
First, $\n\lits2$ is an answer set of
$\reduct{\p}{\n\lits2}$.
So $\n\lits2$ is a minimal (wrt set inclusion)
consistent set of literals which
satisfies $\reduct{\p}{\n\lits2}$.
We shall prove $\n\lits2 \sets \n\lits1$ by showing
$\n\lits1$ also satisfies $\reduct{\p}{\n\lits2}$.
\item
By \ref{reducts_subset}, we have $\reduct{\p}{\n\lits2}
\sets\reduct{\p}{\n\lits1}$.
Because $\n\lits1$ satisfies $\reduct{\p}{\n\lits1}$,
we know $\n\lits1$ also satisfies $\reduct{\p}{\n\lits2}$.
\end{enumerate}

\end{proof}

\begin{definition}
[Equivalence of \cp{} Programs]

Let $\n\p1$ and $\n\p2$ be \cp{} programs.
We say $\n\p1$ and $\n\p2$ are \textdef{equivalent},
denoted by $\n\p1 \equi \n\p2$, when
$\lits$ is an answer set of $\n\p1$ iff $\lits$ is an
answer set of $\n\p2$ for every consistent set of literals
$\lits$.
\end{definition}

\begin{remark}
\label{consistent_p_reg}

Let $\p$ be a \cp{} program.
If $\preg$ is consistent, then $\p$ is equivalent to
$\preg$, which has antichain property due to Theorem
\ref{antichain_a_prolog}.
The interesting case is when $\preg$ is inconsistent.

\end{remark}

\begin{corollary}
\label{one_cr_rule}

Let $\p$ be a \cp{} program where \inconspreg{}.
If $\p$ has only one cr-rule, then $\p$ has
antichain property.

\end{corollary}

\begin{proof}

Assume $\pcr = \set\rl$ for some cr-rule $\rl$.
Then $\p$ is equivalent to the \ap{} program
$\pappld{\set\rl}$.
Simply apply Theorem \ref{antichain_a_prolog} to
$\pappl{\set\rl}$.

\end{proof}

\subsection{Unique CR-Literals}

\begin{definition}
[Useless Regular Rule]

Let $\p$ be an \ap{} program containing some rule $\rl$.
For an answer set $\lits$ of $\p$, we say $\rl$ is
\textdef{useless} wrt $\lits$ if $\lits$ is an answer set
of $\p \setd \set\rl$.

\end{definition}

\begin{definition}
[Regular Rule Firing]

Given a regular rule $\rl$ and a consistent set of literals
$\lits$, we say $\rl$ \textdef{fires} wrt $\lits$ if
$\lits$ satisfies the body of $\rl$.

\end{definition}

\begin{lemma}
\label{unreduced_rule_fires_iff}

Let $\rl$ be a regular rule and $\lits$ be a
consistent set of literals.
Assume $\reduct{\set\rl}\lits = \set{\ot\rl}$ for some
\dnf{} rule $\ot\rl$.
Then $\rl$ fires iff $\ot\rl$ fires, wrt $\lits$.

\end{lemma}

\begin{proof}

The two rules are:
\begin{gather*}
\regrule \tag{$\rl$} \\
\regrulenodefneg \tag{$\ot\rl$}
\end{gather*} \\
By construction of $\ot\rl$ from $\rl$,
we know $\n\lit\ind \not\in \lits$ for all $\ind \in
\set{\nbl + 1, \ldots, \nel}$.
Then $\rl$ fires wrt $\lits$ iff
$\n\lit\ind \in \lits$ for all
$\ind \in \set{\nhl + 1, \ldots, \nbl}$ iff
$\ot\rl$ fires wrt $\lits$.

\end{proof}

\begin{remark}
\label{superset_fires}

Let $\rl$ be a regular \dnf{} rule and
$\n\lits1 \sets \n\lits2$ be consistent sets of literals.
If $\rl$ fires wrt $\n\lits1$, then $\rl$ fires wrt
$\n\lits2$.

\end{remark}

\begin{lemma}
\label{useless_unless_fires}

Let $\p$ be an \ap{} program containing some rule $\rl$.
Let $\lits$ be an answer set of $\p$.
If $\rl$ does not fire, then $\rl$ is useless, wrt $\lits$.

\end{lemma}

\begin{proof}

Let $\n\p1 = \reduct\p\lits$ and
$\n\p2 = \reduct{(\p \setd \set\rl)}\lits$.
We know $\lits$ is an answer set of $\n\p1$.
We will prove $\lits$ is an answer set of $\n\p2$.

\begin{enumerate}

\item
First, $\lits$ satisfies $\n\p1$. Notice
$\n\p2 \sets \n\p1$.
Then $\lits$ satisfies $\n\p2$.

\item
Assume some consistent set of literals
$\ot\lits \sets \lits$ also satisfies $\n\p2$.
We shall show $\ot\lits = \lits$.
\begin{enumerate}
\item
Case 1 of 2: $\reduct{\set\rl}\lits = \sete$.
Then $\n\p1 = \n\p2$, so $\ot\lits$ satisfies $\n\p1$.
\item
Case 2 of 2: $\reduct{\set\rl}\lits = \set{\ot\rl}$ for
some \dnf{} rule $\ot\rl$.
Recall that we hypothesized $\rl$ does not fire wrt
$\lits$. By Lemma \ref{unreduced_rule_fires_iff}, we know
$\ot\rl$ does not fire wrt $\lits$.
Since $\ot\lits \sets \lits$, using the contrapositive of
Remark \ref{superset_fires},
we deduce that $\ot\rl$ does not fire wrt $\ot\lits$.
Then $\ot\lits$ vacuously satisfies $\ot\rl$.
So $\ot\lits$ satisfies $\n\p1 = \n\p2 \setu \set{\ot\rl}$.
\end{enumerate}
In both cases, $\ot\lits$ satisfies $\n\p1$.
As $\lits$ is an answer set of $\n\p1$, due to minimality,
we know $\lits \sets \ot\lits$.
Since we assumed $\ot\lits \sets \lits$, we can now
conclude that $\ot\lits = \lits$.

\item
We have shown that $\lits$ is a minimal consistent set of
literals which satisfies $\n\p2$.
Hence, $\lits$ is an answer set of $\n\p2$.

\end{enumerate}

\end{proof}

\begin{lemma}
\label{cr_rule_not_useless}

Let $\p$ be a \cp{} program.
Let $\lits$ be an answer set of $\p$ and $\rls$ be a
corresponding abductive support.
Let cr-rule $\rl \in \rls$.
Then $\appl\rl$ is not useless wrt $\lits$.

\end{lemma}

\begin{proof}

By way of contradiction, assume $\appl\rl$ is useless wrt
$\lits$.
Let $\pappld\rls$.
Then $\lits$ is an answer set of
$\pappl\rls \setd \set{\appl\rl}$,
whereas this program is precisely $\pappld{\ot\rls}$ with
$\ot\rls = \rls \setd \set\rl$.
So $\pappl{\ot\rls}$ is consistent.
Since $\rls$ is a minimal (wrt cardinality) set of cr-rules
whose applications make $\p$ consistent,
we have $|\rls| \le |\ot\rls|$.
However, $|\ot\rls| = |\rls| - 1$
(recall $\ot\rls = \rls \setd \set\rl$), contradiction.

\end{proof}

\begin{lemma}
\label{abductive_support_fires}

Let $\p$ be a \cp{} program.
Let $\lits$ be an answer set of $\p$ and $\rls$ be a
corresponding abductive support.
Let cr-rule $\rl \in \rls$.
Then the application $\appl\rl$ fires wrt $\lits$.

\end{lemma}

\begin{proof}

First, $\lits$ is an answer set of $\pappld\rls$.
By way of contradiction, assume there exists some cr-rule
$\rl \in \rls$ such that
$\appl\rl$ does not fire wrt $\lits$.
By Lemma \ref{useless_unless_fires}, $\appl\rl$ is useless
wrt $\lits$.
However, by \ref{cr_rule_not_useless}, $\appl\rl$ is not
useless wrt $\lits$, contradiction.

\end{proof}

\begin{definition}
[Head]

If $\rl$ is a rule, then let $\head\rl$ denotes the set of
literals which appear in the head of $\rl$.
If $\rls$ is a set of rules, then let
$\head\rls = \bigcup_{\rl \in \rls} \head\rl$.

\end{definition}

%:supported literal
\begin{definition}
[Supported Literal]

Let $\rl$ be a regular rule,
$\lit$ be a literal,
and $\lits$ be a consistent set of literals.
We say $\lit$ is \textdef{supported} by $\rl$ wrt $\lits$
if all of the following hold:
\begin{enumerate}
\item
$\rl$ fires wrt $\lits$
\item
$\head\rl \seti \lits = \set\lit$
\end{enumerate}
\cite[page 43, Proposition 2.2.1]{krb}

\end{definition}

\begin{proposition}
\label{unique_cr_literals}

Let $\p$ be a \cp{} program.
If the literal in the head of each cr-rule does not appear
in the head of any other rule,
then $\p$ has antichain property.

\end{proposition}

\begin{proof}

Let $\n\lits1$ and $\n\lits2$ be answer sets of $\p$.
Assume $\n\lits1 \sets \n\lits2$.
We will show $\n\lits1 = \n\lits2$.

\begin{enumerate}

\item
Let $\n\rls1$ and $\n\rls2$ respectively be abductive
supports for $\n\lits1$ and $\n\lits2$.
We first prove $\n\rls1 \sets \n\rls2$.
Let $\rl$ be an arbitrary cr-rule in $\n\rls1$.
By Lemma \ref{abductive_support_fires},
$\rl$ fires wrt $\n\lits1$, and
$\n\lits1$ satisfies $\head\rl = \set{\lit}$
where $\lit$ is a literal.
So $\lit \in \n\lits1$, then $\lit \in \n\lits2$
(as $\n\lits1 \sets \n\lits2$).
By the hypothesized syntactic condition, $\rl$ is the sole
rule in $\p$ whose head contains $\lit$.
Then only $\rl$ can support $\lit$ wrt $\n\lits2$.
So $\rl \in \n\rls2$.

\item
Now that we showed $\n\rls1 \sets \n\rls2$,
recall $|\n\rls1| = |\n\rls2|$,
as both are abductive supports.
Then $\n\rls1 = \n\rls2$, since they are finite sets.
Consequently, $\pappld{\n\rls1}$ equals $\pappld{\n\rls2}$.
So $\n\lits1$ and $\n\lits2$ are answer sets of the same
\ap{} program.
By Theorem \ref{antichain_a_prolog}, $\n\lits1 = \n\lits2$.

\end{enumerate}

\end{proof}

%%%%%%%%%%%%%%%%%%%%%%%%%%%%%%%%%%%%%%%%%%%%%%%%%%%%%%%%%%%%

\section{Main Results}

\subsection{Dependency Graphs}

\begin{definition}
[Positive Body of Rule]

Let $\rl$ be a rule.
The \textdef{positive body} of $\rl$,
denoted by $\posbody\rl$,
is the set of literals in the body of $\rl$ which are
\dnf.

\end{definition}

\begin{definition}
[Dependency Graph]

Let $\p$ be a \cp{} program.
Let $\g\p$ be a graph whose vertices are literals of $\p$.
For each pair of vertices $\n\lit1$ and $\n\lit2$ of
$\g\p$, there is a directed edge to $\n\lit1$ from
$\n\lit2$ if $\p$ has some rule $\rl$ where
$\n\lit1 \in \head\rl$ and $\n\lit2 \in \posbody\rl$.
We say $\g\p$ is the \textdef{dependency graph} of $\p$.
Note: a self-loop forms a directed cycle.

\end{definition}

\begin{remark}
[Answer Set of Acyclic Program]
\label{acyclic_answer_set}

Let $\p$ be an acyclic \ap{} program and $\lits$ be a
consistent set of literals.
Then $\lits$ is an answer set of $\p$ iff every literal in
$\lits$ is supported by a rule in $\p$ wrt $\lits$
\cite[page 58, Theorem 2.7]{hcfb}.

\end{remark}

%:ex self-loop
\begin{example}
[Self-Loop Forming Cycle]

To the contrary, assume a self-loop does not form a cycle.
Consider this \ap{} program:
\begin{gather*}
\lit \when \lit. \tag{$\rl$}
\end{gather*}
Let $\lits = \set{\lit}$.
Notice that $\lit$ is supported by $\rl$ wrt $\lits$.
However, $\lits$ is not an answer set of $\p$.
Then Remark \ref{acyclic_answer_set} is wrong,
contradiction.

\end{example}

\begin{definition}
[Head-Cycle]

Let $\p$ be a \cp{} program with dependency graph $\g\p$.
Let $\cy$ be a cycle in $\g\p$.
Assume:
\begin{enumerate}
\item
vertices $\n\lit1$ and $\n\lit2$ are in $\cy$
\item
there exists $\rl \in \p$ where
$\n\lit1$ and $\n\lit2$ are in $\head\rl$
\end{enumerate}
Then $\cy$ is called a \textdef{\hc} in $\p$.

\end{definition}

\begin{definition}
[Dependent Literals]

Let $\p$ be a \cp{} program with dependency graph $\g\p$.
Let $\n\lit1$ and $\n\lit2$ be two literals appearing in
$\p$.
We say $\n\lit1$ \textdef{depends} on $\n\lit2$ if
$\g\p$ has a directed path to $\n\lit1$ from $\n\lit2$.

\end{definition}

\begin{definition}
[CR-Literal]

Let $\p$ be \cp{} program with cr-rule $\rl$ where
$\head\rl = \set\lit$.
We say $\lit$ is a \textdef{cr-literal} of $\p$.

\end{definition}

\begin{definition}
[CR-Independence]

Let $\p$ be a \cp{} program.
Then $\p$ is \textdef{cr-independent} if
$\n\lit1$ does not depend on $\n\lit2$
for all cr-literals $\n\lit1$ and $\n\lit2$.

\end{definition}

\subsection{Same Abductive Support Heads}

\begin{remark}

Let $\p$ be a \cp{} program with some answer set $\lits$.
Then $\lits \sets \head\p$.

\end{remark}

\begin{remark}
\label{empty_reduct_unless_fires}

Let $\rl$ be a regular rule and $\lits$ be a consistent set
of literals.
If $\rl$ fires wrt $\lits$, then
$\reduct{\set\rl}\lits \ne \sete$.

\end{remark}

\begin{remark}
\label{nonempty_reduced_cr_rule}

Let $\p$ be a \cp{} program with some answer set $\lits$
and a corresponding abductive support $\rls$.
Let cr-rule $\rl \in \rls$. Then
$\reduct{\set{\appl\rl}}\lits \ne \sete$.
(Use Lemma \ref{abductive_support_fires} and Remark
\ref{empty_reduct_unless_fires}.)

\end{remark}

\begin{example}

Consider an inconsistent \ap{} program $\p$:
$$\lit \when \negd\lit.$$
Let $\lits = \set\lit$.
Notice $\lits$ is a minimal consistent set of literals
which satisfies $\p$.
However, $\lits$ is not an answer set of $\p$, because
$\lits$ nonminimally satisfies $\reduct\p\lits = \sete$.
(Note that $\ot\lits = \sete$ satisfies neither $\p$ nor
$\reduct\p{\ot\lits} = \set{\lit.}$.)

\end{example}

\begin{example}

Consider a consistent \cp{} program $\p$:
\begin{gather*}
\when \negd {\n\lit1}, \negd {\n\lit2}. \\
\n\lit1 \whencr \negd {\n\lit2}. \tag{$\rl$}
\end{gather*} \\
Then $\rls = \set\rl$ is an abductive support of $\p$.
Indeed, let $\pappld\rls$,
then $\lits = \set{\n\lit1}$ is an answer set of
$\reduct{\pappl\rls}\lits = \set{\n\lit1}$.
Notice $\ot\lits = \set{\n\lit2}$ satisfies $\pappl\rls$,
but $\n\lit1 \not\in \ot\lits$.

\end{example}

\begin{lemma}
\label{intersection_satisfies}

Let $\p$ be a nondisjunctive \dnf{} \ap{} program.
If consistent sets of literals $\n\lits1$ and $\n\lits2$
satisfy $\p$,
then $\n\lits0 = \n\lits1 \seti \n\lits2$ also satisfies
$\p$.

\end{lemma}

\begin{proof}

Let $\rl$ be a rule in $\p$.
Assume $\rl$ fires wrt $\n\lits0$.
Then $\rl$ also fires wrt $\n\lits1$ and $\n\lits2$
(as $\p$ is \dnf).
So $\n\lits1$ and $\n\lits2$ satisfy $\head\rl = \set\lit$
(since $\p$ is nondisjunctive).
Thus $\lit$ is in $\n\lits1$, $\n\lits2$, and $\n\lits0$.

\end{proof}

\begin{lemma}
[Proved by \mg]
\label{answer_set_subprogram}

Let $\p$ be a nondisjunctive \dnf{} \ap{} program.
Assume $\p$ has rules $\n\rl1 \ne \n\rl2$ such that
$\head{\n\rl1} = \head{\n\rl2}$.
Let $\n\p0 = \p \setd \set{\n\rl1, \n\rl2}$,
$\n\p1 = \n\p0 \setu \set{\n\rl1}$,
and $\n\p2 = \n\p0 \setu \set{\n\rl2}$.
If $\lits$ is an answer set of $\p$, then $\lits$ is also
an answer set of either $\n\p1$ or $\n\p2$.

\end{lemma}

\begin{proof}

To the contrary, assume $\lits$ is an answer set of neither
$\n\p1$ nor $\n\p2$.
Still, $\lits$ satisfies both $\n\p1$ and $\n\p2$
(as $\lits$ satisfies $\p$).
So there exist two proper subsets of $\lits$, say
$\n\lits1$ and $\n\lits2$,
which respectively satisfy $\n\p1$ and $\n\p2$
(the programs are \dnf{}).

\begin{enumerate}

\item
Case 1 of 2: either $\n\rl1$ fires wrt $\n\lits1$, or
$\n\rl2$ fires wrt $\n\lits2$.
WOLOG, assume the former. Then $\n\lits1$ satisfies
$\head{\n\rl1} = \head{\n\rl2}$.
So $\n\lits1$ also satisfies both the rule $\n\rl2$ and the
program $\p = \n\p1 \setu \set{\n\rl2}$.
Recall that $\lits$ is an answer set of $\p$ whereas
$\n\lits1 \setsp \lits$, contradiction.

\item
Case 2 of 2: neither $\n\rl1$ fires wrt $\n\lits1$, nor
$\n\rl2$ fires wrt $\n\lits2$.
So neither $\n\rl1$ nor $\n\rl2$ fires wrt
$\n\lits0 = \n\lits1 \seti \n\lits2$
(the rules are \dnf).
Then $\n\lits0$ vacuously satisfies $\n\rl1$ and $\n\rl2$.
Because $\n\lits1$ and $\n\lits2$ satisfy $\n\p0$,
by \ref{intersection_satisfies},
$\n\lits0$ satisfies $\n\p0$ too.
Therefore, $\n\lits0$ satisfies
$\p = \n\p0 \setu \set{\n\rl1, \n\rl2}$.
But $\lits$ is an answer set of $\p$, and
$\n\lits0 \setsp \lits$, contradiction.

\end{enumerate}

\end{proof}

%:diff cr-literals
\begin{lemma}
\label{diff_cr_literals}

Let $\p$ be a nondisjunctive \cp{} program with some
abductive support $\rls$.
Then for all cr-rules $\n\rl1$ and $\n\rl2$ in $\rls$,
we have
$\head{\n\rl1} = \head{\n\rl2} \impl \n\rl1 = \n\rl2$.

\end{lemma}

\begin{proof}

By way of contradiction, assume there exist cr-rules
$\n\rl1\ne \n\rl2$ in $\rls$ where
$\head{\n\rl1} = \head{\n\rl2}$.
Some notations:
\begin{enumerate}
\item
Let $\n\rls1 = \rls \setd \set{\n\rl2}$ and
$\n\rls2 = \rls \setd \set{\n\rl1}$.
\item
Let $\n\p1 = \applp{\n\rls1}$ and
$\n\p2 = \applp{\n\rls2}$.
\item
Let $\lits$ be an answer set of $\p$ wrt $\rls$.
\item
Let $\n\p a = \reduct{\n\p1}\lits$ and
$\n\p b = \reduct{\n\p2}\lits$.
Note that by \ref{nonempty_reduced_cr_rule},
$\reduct{\set{\appl{\n\rl1}}}\lits = \set{\ot{\n\rl1}}$ and
$\reduct{\set{\appl{\n\rl2}}}\lits = \set{\ot{\n\rl2}}$ for
some \dnf{} rules $\ot{\n\rl1}$ and $\ot{\n\rl2}$.
\end{enumerate}

By \ref{answer_set_subprogram}, $\lits$ is an answer set of
either $\n\p a$ or $\n\p b$.
WOLOG, assume the former.
Then $\lits$ is an answer set of $\n\p1$.
So $\n\rls1$ is an abductive support of $\p$,
but $|\n\rls1| < |\rls|$, contradiction.

\end{proof}

%:sole supporting rule
\begin{lemma}
\label{sole_supporting_rule}

Let $\p$ be an acyclic \cp{} program having some answer set
$\lits$ wrt an abductive support $\rls$.
Let cr-rule $\rl \in \rls$ where $\head\rl = \set\lit$.
Then $\appl\rl$ is the only rule in $\pappld\rls$ which
supports $\lit$ wrt $\lits$.

\end{lemma}

\begin{proof}

By way of contradiction, assume $\lit$ is also supported by
a rule $\ot\rl \ne \appl\rl$ in $\pappl\rls$.
Let $\ot\rls = \rls \setd \set{\rl}$ and
$\ot\p = \applp{\ot\rls}$.
We will prove $\lits$ is an answer set of $\ot\p$:
\begin{enumerate}
\item
First, $\lits$ satisfies $\ot\p \sets \pappl\rls$.
\item
Let $\ot\lit$ be an arbitrary literal in $\lits$. We show
$\ot\lit$ is supported wrt $\lits$ by some rule in $\ot\p$.
Recall that $\lits$ is an answer set of $\pappl\rls$.
By \ref{acyclic_answer_set}, $\ot\lit$ is supported wrt
$\lits$ by some rule $\n\rl0$ in $\pappl\rls$.
\begin{enumerate}
\item
Case 1 of 2: $\n\rl0 = \appl\rl$.
Recall $\head\rl = \set\lit$.
Then $\ot\lit = \lit$.
Notice $\ot\rl$ also supports $\ot\lit$ wrt $\lits$, and
$\ot\rl \in \ot\p$.
\item
Case 2 of 2: $\n\rl0 \ne \appl\rl$.
Then $\n\rl0 \in \ot\p$.
\end{enumerate}
\end{enumerate}
So $\ot\p$ is consistent, and $\ot\rls$ is an abductive
support of $\p$.
But $|\ot\rls| < |\rls|$, contradiction.

\end{proof}

\begin{definition}
[Factified Rule]

Let $\rl$ be a regular rule.
Let $\factify\rl$ denote the \textdef{factified rule}
obtained from $\rl$ by dropping its body.
If $\rls$ is a set of rules, then let
$\factify\rls = \set{\factify\rl : \rl \in \rls}$.

\end{definition}

%:factify answer set
\begin{lemma}
\label{factify_answer_set}

Let $\p$ be a \cp{} program with answer set $\lits$ and
corresponding abductive support $\rls$.
Then $\lits$ is an answer set of
$\ot\p = \preg \setu \factify{\appl\rls}$.

\end{lemma}

\begin{proof}
We prove $\lits$ is a minimal consistent set of literals
which satisfies $\reduct{\ot\p}\lits$:
\begin{enumerate}
\item
Let $\pappld\rls$.
Note that $\head\rls \sets \lits$ as $\lits$ is an answer
set of $\pappl\rls$.
Then $\lits$ satisfies
$\reduct{\ot\p}\lits =
\reduct{(\preg)}{\lits} \setu \factify{\appl\rls}$.
\item
Assume some $\ot\lits \sets \lits$ also satisfies
$\reduct{\ot\p}\lits$.
We know $\head\rls \sets \ot\lits$.
Then $\ot\lits$ satisfies
$\reduct{(\pappl\rls)}\lits =
\reduct{(\preg)}\lits \setu \reduct{(\appl\rls)}\lits$.
So $\lits \sets \ot\lits$
(as $\lits$ is an answer set of
$\pappl\rls$). Therefore $\ot\lits = \lits$.
\end{enumerate}

\end{proof}

%:same abductive support heads
\begin{proposition}
\label{same_abductive_support_heads}

Let $\p$ be a \cp{} program.
Then $\p$ has antichain property if
$\head{\n\rls1} = \head{\n\rls2}$ for
all abductive supports $\n\rls1$ and $\n\rls2$ of $\p$.

\end{proposition}

\begin{proof}

Let $\n\lits1$ and $\n\lits2$ be arbitrary answer sets of
$\p$ corresponding to
some abductive supports $\n\rls1$ and $\n\rls2$.
By \ref{factify_answer_set}, $\n\lits1$ and $\n\lits2$ are
respectively answer sets of
$\preg \setu \factify{\appl{\n\rls1}}$ and
$\preg \setu \factify{\appl{\n\rls2}}$,
which are the same \ap{} program because
$\head{\n\rls1} = \head{\n\rls2}$.
By Theorem \ref{antichain_a_prolog},
$\n\lits1 \sets \n\lits2 \impl \n\lits1 = \n\lits2$.

\end{proof}

\subsection{CR-Independence and Acyclicity}

%:proof of literal
\begin{definition}
[Proof of Literal]

Let $\p$ be an \ap{} program,
$\lits$ be a consistent set of literals,
and $\lit$ be a literal.
A \textdef{proof} of $\lit$ wrt $\lits$ in $\p$ is
a nonempty sequence of rules
$\pr = \n\rl1, \ldots, \n\rl n$ in $\p$ such that:
\begin{enumerate}
\item
for each rule $\n\rl i$, there is a (sole) literal
$\ot\lit$ supported by $\rl$ wrt $\lits$;
let $h_\lits(\rl)$ denote $\ot\lit$
\item
$\lit = h_\lits(\n\rl n)$
\item
$\posbody{\n\rl1} = \sete$
%:hcf error
\footnote
{Their original definition states: $\n\rl1$ is a fact.
However, there is an error in their Lemma B.5 on page 83.
Consider an acyclic \ap{} program $\set{a \when \negd b.}$
with answer set $\set{a}$.}
\item
for each rule $\n\rl i$,
every literal in $\posbody{\n\rl i}$ is $h_\lits(\n\rl j)$
for some $j < i$
\end{enumerate}
\cite[page 57]{hcfb}

\end{definition}

\begin{definition}

Additional notations:
\begin{enumerate}
\item
$\prs{\lit}{\lits}{\p}$ denotes the set of all proofs $\pr$
of $\lit$ wrt $\lits$ in $\p$
\item
$h_\lits(\pr)$ denotes $\set{h_\lits(\rl) : \rl \in \pr}$
\item
$\posbody\pr$ denotes $\set{\posbody\rl : \rl \in \pr}$
\end{enumerate}

\end{definition}

\begin{remark}

Let $\pr = \rlseq1n \in \prs\lit\lits\p$.
We adopt the convention that the rules in $\pr$ are
pairwise distinct.
Indeed, if there were $\n\rl i = \n\rl j$ where
$i < j$, then $\ot\pr =
\rlseq1{j - 1}, \rlseq{j + 1}n \in \prs\lit\lits\p$,
so $\n \rl j$ would be redundant.

\end{remark}

\begin{remark}

Let $\p$ be a \hcf{} \ap{} program with answer set $\lits$.
Then each literal in $\lits$ has a proof wrt $\lits$ in
$\p$ \cite[page 57]{hcfb}.

\end{remark}

\begin{definition}
[Rank of Literal]

Let $\p$ be a \hcf{} \ap{} program and
$\lits$ be a consistent set of literals.
The \textdef{rank} of a literal $\lit$ wrt $\lits$ in $\p$,
denoted by $\rkt\lit\lits\p$, is
$min\set{|\pr| : \pr \in \prs\lit\lits\p}$.

\end{definition}

\begin{definition}
[Ranking Function]

Let:
\begin{enumerate}
\item
$\p$ be a \hcf{} \ap{} program
\item
$\lits$ be an answer set of $\p$
\item
function $\rkn : \lits \to \mathbb{N}$
\end{enumerate}

Then $\rkn$ is the \textdef{ranking function}
wrt $\lits$ in $\p$ if $\rku\lit = \rkt\lit\lits\p$
for all $\lit \in \lits$.

\end{definition}

\begin{definition}
[Normal Proof of Literal]

Let:
\begin{enumerate}
\item
$\p$ be a \hcf{} \ap{} program
\item
$\lits$ be an answer set of $\p$
\item
$\lit \in \lits$
\item
$\pr \in \prs\lit\lits\p$
\item
$\rkn$ be the ranking function wrt $\lits$ in $\p$
\end{enumerate}

We say $\pr$ is a \textdef{normal proof} if
for all $\rl \in \pr$ and $\ot\lit \in \posbody\rl$,
we have $\rku{\ot\lit} < \rku{h_\lits(\rl)}$.

\end{definition}

\begin{remark}

Let:
\begin{enumerate}
\item
$\p$ be a \hcf{} \ap{} program
\item
$\lits$ be an answer set of $\p$
\item
$\rkn$ be the ranking function wrt $\lits$ in $\p$
\item
$\lit \in \lits$
\item
normal $\pr = \rlseq1n \in \prs\lit\lits\p$
\item
$\n\rl i \in \pr$
\item
$\n\lit i \in \posbody{\n\rl i}$.
\end{enumerate}

Then:
\begin{enumerate}
\item
$\rku{\n\lit i} < \rku\lit$
\item
$\rlseq1i$ is a normal proof of $h_\lits(\n\rl i)$
\end{enumerate}

\end{remark}

%:ex nonminimal subproof
\begin{example}
[Nonminimal Subproof of Minimal Proof]

Consider this acylic \ap{} program $\p$:
\begin{gather}
a \when b, c. \\
b \when c1x. \\
c \when c1x. \\
c1x \when c1y. \\
c1y. \\
c \when c2. \\
c2.
\end{gather} \\
The sole answer set of $\p$ is
$\lits = \set{a, b, c, c1x, c1y, c2}$.
The only minimal proofs of literal $a$ wrt
$\lits$ in $\p$ are
the two sequences of rules $<5, 4, 3, 2, 1>$ and
$<5, 4, 2, 3, 1>$.
In both these proofs, the subproof of $c$ is $<5, 4, 3>$,
which is nonminimal.
(The minimal proof of $c$ wrt $\lits$ in $\p$ is $<7, 6>$.)

\end{example}

%:rule in P2
\begin{lemma}
\label{rule_in_P2}

Let:
\begin{enumerate}
\item
$\n\p1$ and $\n\p2$ be \hcf{} \ap{} programs with
corresponding answer sets $\n\lits1 \setsp \n\lits2$
\item
$\lit \in \n\lits2 \setd \n\lits1$
\end{enumerate}

Then for each normal
$\pr = \rlseq1n \in \prs\lit{\n\lits2}{\n\p2}$,
there exists $\rl \in \pr$ such that
$\rl \in \n\p2 \setd \n\p1$.

\end{lemma}

\begin{proof}

Let $\rkn$ be the ranking function
wrt $\n\lits2$ in $\n\p2$.
We employ induction on $\rku\lit$.

\begin{enumerate}

\item
Base step: $\rku\lit = min\set{\rku{\ot\lit} :
\ot\lit \in \n\lits2 \setd \n\lits1}$.
\begin{enumerate}
\item
To the contrary, assume for every $\rl \in \pr$, we have
$\rl \in \n\p1 \seti \n\p2$.
\item
Then $\n\rl n \in \n\p1$.
\item
As $\pr$ is normal, for each
$\ot\lit \in \posbody{\n\rl n}$,
we have $\rku{\ot\lit} < \rku\lit$,
so $\ot\lit \in \n\lits1$.
\item
Then $\n\rl n$ fires wrt $\n\lits1$.
\item
As $\n\lits1$ is an answer set of $\n\p1$, we know
$\n\lits1$ satisfies $\head{\n\rl n}$.
\item
Let $\ot\lit \in \head{\n\rl n}$.
\begin{enumerate}
\item
Case 1 of 2: $\ot\lit = \lit$.
We already assumed $\lit \in \n\lits2 \setd \n\lits1$.
\item
Case 2 of 2: $\ot\lit \ne \lit$.
We have $\ot\lit \not\in \n\lits2$,
so $\ot\lit \not\in \n\lits1$.
\end{enumerate}
\item
So $\n\lits1$ does not satisfy $\head{\n\rl n}$,
contradiction.
\end{enumerate}
\item
Induction step: $\rku\lit \le max\set{\rku{\ot\lit} :
\ot\lit \in \n\lits2 \setd \n\lits1}$.
\begin{enumerate}
\item
Induction hypothesis: for each
$\ot\lit \in \n\lits2 \setd \n\lits1$,
let normal $\ot\pr \in \prs{\ot\lit}{\n\lits2}{\n\p2}$,
if $\rku{\ot\lit} < \rku\lit$,
then there exists ${\rl} \in \ot\pr$
such that ${\rl} \in \n\p2 \setd \n\p1$.
\item
To the contrary, assume for every $\rl \in \pr$, we have
$\rl \in \n\p1 \seti \n\p2$.
\begin{enumerate}
\item
Case 1 of 2: there exists $\ot\lit \in \posbody{\n\rl n}$
where $\ot\lit \in \n\lits2 \setd \n\lits1$.
\begin{enumerate}
\item
Choose some $1 \le m \le n$ where
$h_{\n\lits2}(\n\rl m) = \ot\lit$.
\item
As $\pr$ is normal, $\ot\pr = \n\rl1, \ldots, \n\rl m$ is a
normal proof of
$\ot\lit$ wrt $\n\lits2$ in $\n\p2$.
\item
Notice $\rku{\ot\lit} < \rku\lit$.
\item
By the induction hypothesis, $\ot\pr$ contains some
${\rl} \in \n\p2 \setd \n\p1$.
\item
So $\pr$ also contains ${\rl}$.
\item
By our assumption, $\rl \in \n\p1 \seti \n\p2$,
contradiction.
\end{enumerate}
\item
Case 2 of 2: for every $\ot\lit \in \posbody{\n\rl n}$,
we have $\ot\lit \in \n\lits1$.
\begin{enumerate}
\item
Then $\n\rl n$ fires wrt $\n\lits1$.
\item
By our assumption, $\n\rl n \in \n\p1$.
\item
As $\n\lits1$ is an answer set of $\n\p1$, we know
$\n\lits1$ satisfies $\head{\n\rl n}$.
\item
But $\head{\n\rl n} \seti \n\lits1 = \sete$, contradiction.
\end{enumerate}
\end{enumerate}
\end{enumerate}

\end{enumerate}

\end{proof}

\begin{remark}

Let $\pr = \n\rl1, \ldots, \n\rl n \in \prs\lit\lits\p$. If
$\pr$ is minimal, then:
\begin{enumerate}
\item
$\pr$ is normal
\item
$\lit$ depends on $h_\lits(\n\rl i)$ for all $i < n$
\end{enumerate}

\end{remark}

%:last lemma
\begin{lemma}
\label{last_lemma}

Let $\p$ be a nondisjunctive acyclic \cp{} program.
If $\p$ has answer sets $\n\lits1 \setsp \n\lits2$,
then there exist literals $\n\lit1$ and $\n\lit2$ in
$\head\pcr$ such that
$\n\lit1$ depends on $\n\lit2$.

\end{lemma}

\begin{proof}

Notations:
\begin{enumerate}
\item
By the contrapositive of Proposition \ref{same_abductive_support_heads},
there exist abductive supports
$\n\rls1$ and $\n\rls2$ of $\p$ where
$\head{\n\rls1} \ne \head{\n\rls2}$.
\item
Let $\ot{\n\rls1} = \factify{\appl{\n\rls1}}$ and
$\ot{\n\rls2} = \factify{\appl{\n\rls2}}$.
\item
Let $\n\p1 = \preg \setu \ot{\n\rls1}$ and
$\n\p2 = \preg \setu \ot{\n\rls2}$.
By Lemma \ref{factify_answer_set},
$\n\lits1$ and $\n\lits2$ are respectively answer sets of
$\n\p1$ and $\n\p2$.
\end{enumerate}

We follow these steps:
\begin{enumerate}
\item
By Lemma \ref{diff_cr_literals},
since $\p$ is nondisjunctive, we have
$|\ot{\n\rls1}| = |\n\rls1|$ and
$|\ot{\n\rls2}| = |\n\rls2|$.
\item
Notice $|\n\rls1| = |\n\rls2| > 0$.
\item
Then $|\ot{\n\rls1}| = |\ot{\n\rls2}| > 0$.
\item
Observe $|\head{\ot{\n\rls1}}| = |\ot{\n\rls1}|$ and
$|\head{\ot{\n\rls2}}| = |\ot{\n\rls2}|$.
\item
Thus $|\head{\ot{\n\rls1}}| = |\head{\ot{\n\rls2}}| > 0$.
\item
Recall $\head{\ot{\n\rls1}} = \head{\n\rls1} \ne
\head{\n\rls2} = \head{\ot{\n\rls2}}$.
\item
Then $\head{\ot{\n\rls1}} \setd \head{\ot{\n\rls2}} \ne
\sete$.
\item
Select some $\n\lit1 \in \head{\ot{\n\rls1}} \setd
\head{\ot{\n\rls2}}$.
\item
Let $\n\rl1$ be the fact $(\n\lit1.)$ in
$\ot{\n\rls1} \setd\ot{\n\rls2}$.
\item
Since $\n\lits1$ is an answer set of $\n\p1$, we have
$\n\lit1 \in \n\lits1$.
\item
Recall $\n\lits1 \setsp \n\lits2$.
Then $\n\lit1 \in \n\lits2$.

\item
As $\n\lits2$ is an answer set of $\n\p2$, there exists
$\rl\in \n\p2$ which
supports $\n\lit1$ wrt $\n\lits2$.
\begin{enumerate}
\item
Case 1 of 2: there exists $\lit \in \posbody{\rl}$ where
$\lit \in \n\lits2 \setd \n\lits1$.
\begin{enumerate}
\item
Let minimal $\pr \in \prs\lit{\n\lits2}{\n\p2}$.
\item
By Lemma \ref{rule_in_P2}, there exists $\n\rl2 \in \pr$
where $\n\rl2 \in \n\p2 \setd \n\p1$.
\item
Then $\n\rl2 \in \ot{\n\rls2} \setd \ot{\n\rls1} \sets
\appl\pcr$.
\item
Let $h_{\n\lits2}(\n\rl2) = {\n\lit2} \in \head\pcr$.
\item
As $\pr$ is minimal, $\lit$ depends on $\n\lit2$.
\item
Recall $\n\lit1$ depends on $\lit$ in $\rl$.
\item
By transitivity, $\n\lit1$ depends on $\n\lit2$.
\end{enumerate}
\item
Case 2 of 2: $\posbody{\rl} \sets \n\lits1$.
We show this case is impossible.
\begin{enumerate}
\item
Subcase 1 of 2: $\rl \in \n\p1 \seti \n\p2$.
\begin{enumerate}
\item
Notice $\rl$ supports $\n\lit1$ wrt $\n\lits1$.
\item
By Lemma \ref{sole_supporting_rule}, since $\p$ is acyclic,
we have $\rl = \n\rl1$.
\item
However, $\rl \in \n\p2$ whereas
$\n\rl1 \in \n\p1 \setd \n\p2$, contradiction.
\end{enumerate}
\item
Subcase 2 of 2: $\rl \in \n\p2 \setd \n\p1$.
\begin{enumerate}
\item
So $\rl \in \ot{\n\rls2} \setd \ot{\n\rls1}$.
\item
Then $\rl$ is the fact $(\n\lit1.)$,
which is exactly $\n\rl1$.
\item
However, we selected $\n\rl1$ from
$\ot{\n\rls1} \setd \ot{\n\rls2}$
while $\rl \in \ot{\n\rls2}$, contradiction.
\end{enumerate}
\end{enumerate}
\end{enumerate}
\end{enumerate}

\end{proof}

%:hcf nondisjunctive
\begin{remark}
\label{hcf_nondisjunctive}

If a \cp{} program $\p$ is acyclic,
then there exists a nondisjunctive acyclic \cp{} program
$\ot\p$ equivalent to $\p$
\cite[page 73, Theorem 4.17]{hcfb}.

\end{remark}

%:last theorem
\begin{theorem}
[Antichain Property of CR-Independent Acyclic
\cp{} Programs]
\label{last_theorem}

Let $\p$ be an acyclic \cp{} program.
If $\p$ is cr-independent,
then $\p$ has antichain property.

\end{theorem}

\begin{proof}

By Remark \ref{hcf_nondisjunctive},
there exists a nondisjunctive acyclic \cp{} $\ot\p$
equivalent to $\p$.
By the contrapositive of Lemma \ref{last_lemma},
$\ot\p$ has antichain property.
Therefore, $\p$ has antichain property too.

\end{proof}

%%%%%%%%%%%%%%%%%%%%%%%%%%%%%%%%%%%%%%%%%%%%%%%%%%%%%%%%%%%%

\section{Conclusion}

We have found a reasonably weak syntactic condition
which guarantees that
a \cp{} program has antichain property:
the program being acyclic and cr-independent.
The future goal is to further relax
this sufficient condition to extend the class of
\cp{} programs known to have
this desired semantic property.

%%%%%%%%%%%%%%%%%%%%%%%%%%%%%%%%%%%%%%%%%%%%%%%%%%%%%%%%%%%%

\section{Acknowledgment}

We sincerely appreciate the significant guidance of \mg.
This paper includes his proofs of
Theorem \ref{antichain_a_prolog} and
Lemma \ref{answer_set_subprogram}.
We also thank Evgenii Balai for the
informative discussions.

%%%%%%%%%%%%%%%%%%%%%%%%%%%%%%%%%%%%%%%%%%%%%%%%%%%%%%%%%%%%

\end{flushleft}

\newpage

\bibliography{bibl}

\end{document}
